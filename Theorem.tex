\documentclass[12pt]{article}

\usepackage{lingmacros}
\usepackage{tree-dvips}
\usepackage{amsmath}
\usepackage{amsfonts}
\usepackage{hyperref}
\usepackage[english,ukrainian]{babel}
\usepackage{amsthm}

\newtheorem{theorem}{Теорема}

\title{Integral equation method for inverse boundary value problem for biharmonic equation}
\date{}

\begin{document}
\maketitle

% 1 // Формулювання %

\section {Формулювання}
Нехай $\Omega\subset \mathbb{R}^2$ двоз'язна область з межею $\Gamma=\Gamma_1\cup\Gamma_2$, де $\Gamma_1$ є внутрішньою межею and $\Gamma_2$ - зовнішньою. І нехай $\Gamma_1$ невідома.

\begin{equation}
	\left\{
	\label{directProblem}
	\begin{split}
		&\Delta^2 u(x)=0, \quad x\in\Omega, \\
		&u(x)=\frac{\partial u(x)}{\partial n}=0, \quad x\in\Gamma_1, \\
		&\frac{\partial u(x)}{\partial n}=g(x), \quad x\in\Gamma_2, \\
		&Mu(x)=q(x), \quad x\in\Gamma_2,
	\end{split}
	\right.
\end{equation}

\begin{equation}
	\label{dataEq}
	u(x)=f(x), \quad x\in\Gamma_2,
\end{equation}
де $\Delta^{2}u(x)=\frac{\partial^{4}}{\partial x_1^4}u(x)+2\frac{\partial^{4}}{\partial x_1^2\partial x_2^2}u(x)+\frac{\partial^{4}}{\partial x_2^4}u(x)=0,\ u(x)=u(x_1,x_2), \ n=(n_1, n_2)$ - зовнішня нормаль до $\Gamma$, $Mu=\nu\Delta u+(1-\nu)(u_{x_1 x_1}u_1^2+2u_{x_1 x_2}n_1 n_2+u_{x_2 x_2}u_2^2), \ \nu\in (0, 1)$ і $g(x),q(x),f(x)$ - деякі задані функції. Роз'язування \eqref{directProblem}-\eqref{dataEq} складається зі знаходження невідомої $\Gamma_1$ для заданих граничних умов.

% 2 // Теореми %

\section {Загальні положення}

Фундаментальний роз'вязок бігармонійного рівняння має вигляд
 \begin{equation}
 	G(x, y)=\frac{1}{8\pi}|x-y|^2\ln|x-y|, \quad x,\ y \in \mathbb{R}^2.
 \end{equation}

Розглянемо потенціал простого шару для бігармонійного рівняння з густинами $\varphi$, $\psi$, що визначені на $\Gamma$:

\begin{equation}
	 	u(x)=\int_{\Gamma}\bigg(G(x,y)\varphi(y)+\frac{\partial G(x,y)}{\partial n_y}\psi(y)\bigg)d\sigma_y, \quad x\in \Omega. \nonumber
\end{equation}

\begin{theorem}
Розв'язок прямої крайової задачі \eqref{directProblem} має вигляд
	 \begin{equation}
	 	u(x)=\sum_{k=1}^{2}\int_{\Gamma_k}\bigg(G(x,y)\varphi_k(y)+\frac{\partial G(x,y)}{\partial n_y}\psi_k(y)\bigg)d\sigma_y+\omega(x), \quad x\in \Omega,
	 \end{equation}
	 де $\omega(x) = \alpha_0+\alpha_1x_1+\alpha_2x_2 \ ((\alpha_0,\alpha_1,\alpha_2)\in R^3), \varphi_k,\psi_k\in C(\Gamma_k), k=1,2,$, де невідомі густини визначаються із системи інтегральних рівнянь
	 \begin{equation}
	 \left\{
	 	\begin{split}
		\label{system}
	 		&\sum_{k=1}^{2}\int_{\Gamma_k}\bigg(G(x,y)\varphi_k(y)+\frac{\partial G(x,y)}{\partial n_y}\psi_k(y)\bigg)d\sigma_y+\omega(x)=0, \ x\in\Gamma_1, \\
			&\sum_{k=1}^{2}\int_{\Gamma_k}\bigg(\frac{\partial G(x,y)}{\partial n_x}\varphi_k(y)+\frac{\partial^2 G(x,y)}{\partial n_y\partial n_x}\psi_k(y)\bigg)d\sigma_y+\frac{\partial\omega(x)}{\partial n}=0, \ x\in\Gamma_1, \\
			&\sum_{k=1}^{2}\int_{\Gamma_k}\bigg(\frac{\partial G(x,y)}{\partial n_x}\varphi_k(y)+\frac{\partial^2 G(x,y)}{\partial n_y\partial n_x}\psi_k(y)\bigg)d\sigma_y+\frac{\partial\omega(x)}{\partial n}=g(x), \ x\in\Gamma_2, \\
			&\sum_{k=1}^{2}\int_{\Gamma_k}\bigg(M_x G(x,y)\varphi_k(y)+\frac{\partial M_x G(x,y)}{\partial n_y}\psi_k(y)\bigg)d\sigma_y+M_x\omega(x)=q(x), \ x\in\Gamma_2, \\
			&\sum_{k=1}^{2}\int_{\Gamma_k}\varphi_k(y)d\sigma_y=A_0, \\
			&\sum_{k=1}^{2}\int_{\Gamma_k}(y_1\varphi_k(y)+n_1(y)\psi_k(y))d\sigma_y=A_1, \\
			&\sum_{k=1}^{2}\int_{\Gamma_k}(y_2\varphi_k(y)+n_2(y)\psi_k(y))d\sigma_y=A_2,
		\end{split}
	\right.
	 \end{equation}
	для заданих $(A_0,A_1,A_2)\in R^3$.
 \end{theorem}

З умови \eqref{dataEq} маємо

 \begin{equation}
	\label{eq}
	 \sum_{k=1}^{2}\int_{\Gamma_k}\bigg(G(x,y)\varphi_k(y)+\frac{\partial G(x,y)}{\partial n_y}\psi_k(y)\bigg)d\sigma_y+\omega(x)=f(x), \ x\in\Gamma_2, 
\end{equation}

\begin{theorem}
Обернена крайова задача \eqref{directProblem}-\eqref{dataEq} еквівалентна системі інтегральних рівнянь \eqref{system}-\eqref{eq}.
\end{theorem}

\begin{theorem}[Існування і єдиність розв'язку непрямої задачі]

Нехай $\tilde{\Gamma}_1$, $\Gamma_1$ - замкнені криві, що містяться всередині $\Gamma_2$, $\tilde{u}$, $u$ - розв'язки задачі $\eqref{directProblem}$ для $\tilde{\Gamma}_1$ і $\Gamma_1$ відповідно. Нехай $g\neq 0, q\neq 0$ i $u = \tilde{u}$ на відкритій підмножині $\Gamma_2$. Тоді $\tilde{\Gamma}_1=\Gamma_1$.

\end{theorem}

\begin{proof}

Доведемо від протилежного. Нехай $\tilde{\Gamma}_1\neq\Gamma_1$.
Нехай $\Omega_2$ - область, обмежена $\Gamma_2$, $\Omega_1$, $\tilde{\Omega}_1$ - області обмежені $\Gamma_1$, $\tilde{\Gamma}_1$ відповідно. 
\begin{equation}
W:=\Omega_2\setminus(\Omega_1\cup\tilde{\Omega}_1), \quad \Gamma_2\subset\delta W. \nonumber
\end{equation}
За теоремою Гольмгрена маємо $u = \tilde{u}$ в $W$. Без втрати загальності приспустимо, що $W^{*}:=(\Omega_2\setminus \overline{W})\setminus\Omega_1$ непорожня множина. Тоді $u$ є визначена в $W^{*}$ як розв'язок задачі \eqref{directProblem} для $\Gamma_1$. Вона є гармонічною в $W^{*}$, неперервною в $\overline{W}^{*}$ і задовольняє граничні умови на $\delta W^{*}$. Ця гранична умова випливає з того, що кожна точка з $W^{*}$ належить або $\Gamma_1$, або $\delta W\cap\tilde{\Gamma}_1$. Для $x\in\Gamma_1$ маємо $u(x)=0$ як наслідок граничних умов для $u$, для $x\in\tilde{\Gamma}_1$ маємо $u(x)=\tilde{u}(x)$ і тому $u(x)=0$ як наслідок граничних умов для $\tilde{u}$. Тоді за принципом максимуму $u=0$ в $W^{*}$ і тому $u=0$ в $D$. Це суперечить тому, що $g\neq 0, q\neq 0$.
\end{proof}

% 3 // Параметризація %

\section {Параметризація}

Припустимо, що дані криві $\Gamma_1$ i $\Gamma_2$ достатньо гладкі і їх можна подати у параметричному заданні 
 
 \begin{equation}
 	\Gamma_l=\left\{x_l(s)=(x_1{_l}(s),x_2{_l}(s)) \ : \ s\in [0,2\pi]\right\},
  \end{equation}
 де $x_l \ (l=1,2)$ - аналітична й $2\pi$-періодична функція, $|x'(s)|>0.$
 Тоді систему можна записати 
 
 \begin{equation}
 		\left\{
	 	\begin{split}
		\label{paramSystem}
	 		&\frac{1}{2\pi}\sum_{k=1}^{2}\int_{0}^{2\pi}\bigg(H_1{_k}(s, \sigma)\varphi_k(\sigma)+\tilde{H}_1{_k}(s, \sigma)\psi_k(\sigma)\bigg)d\sigma+\omega(x_1(s))=0,\\
			&\frac{1}{2\pi}\sum_{k=1}^{2}\int_{0}^{2\pi}\bigg(\tilde{\tilde{H}}_1{_k}(s, \sigma)\varphi_k(\sigma)+\hat{H}_1{_k}(s, \sigma)\psi_k(\sigma)\bigg)d\sigma+\frac{\partial\omega(x_1(s))}{\partial n_1}=0, \\
			&\frac{1}{2\pi}\sum_{k=1}^{2}\int_{0}^{2\pi}\bigg(\tilde{\tilde{H}}_2{_k}(s, \sigma)\varphi_k(\sigma)+\hat{H}_2{_k}(s, \sigma)\psi_k(\sigma)\bigg)d\sigma+\frac{\partial\omega(x_2(s))}{\partial n_2}=g(x_2(s)), \\
			&\frac{1}{2\pi}\sum_{k=1}^{2}\int_{0}^{2\pi}\bigg(M_x H_2{_k}(s, \sigma)\varphi_k(\sigma)+M_x\tilde{H}_2{_k}(s, \sigma)\psi_k(\sigma)\bigg)d\sigma+M_x\omega(x_2(s))=q(x_2(s)),\\
			&\sum_{k=1}^{2}\int_{0}^{2\pi}\varphi_k(\sigma)d\sigma=A_0, \\
			&\sum_{k=1}^{2}\int_{0}^{2\pi}(x_1{_k}\varphi_k(\sigma)+n_1(x_k(\sigma))\psi_k(\sigma))d\sigma=A_1, \\
			&\sum_{k=1}^{2}\int_{0}^{2\pi}(x_2{_k}\varphi_k(\sigma)+n_2(x_k(\sigma))\psi_k(\sigma))d\sigma=A_2, \\
		\end{split}
		\right.
\end{equation}

 \begin{equation}
	 \frac{1}{2\pi}\sum_{k=1}^{2}\int_{0}^{2\pi}\bigg(H_2{_k}(s, \sigma)\varphi_k(\sigma)+\tilde{H}_2{_k}(s, \sigma)\psi_k(\sigma)\bigg)d\sigma+\omega(x_2(s))=f(x_2(s)), 
\end{equation}

для $s\in [0,2\pi]$.

Тут $ \label{kernels} \varphi_l(s) :=\varphi_k(x_l(s))|x'_l(s)|, \ \psi_l(s) :=\psi_k(x_l(s))|x'_l(s)| \ - \ \textrm{невідомі густини} $ і ядра мають вигляд
 
 \begin{equation}
 \begin{split}
	&H_l{_k}(s, \sigma) = G(x_l(s),x_k(\sigma)), \quad \tilde{H}_l{_k}(s, \sigma)=\frac{\partial G(x_l(s),x_k(\sigma))}{\partial n_y}, \\
	&\tilde{\tilde{H}}_l{_k}(s, \sigma)=\frac{\partial G(x_l(s),x_k(\sigma))}{\partial n_x}, \quad \hat{H}_l{_k}(s, \sigma)=\frac{\partial^2 G(x_l(s),x_k(\sigma))}{\partial n_y\partial n_x}, \\
	&n(x(s))=\Big(\frac{x'_2(s)}{|x'(s)|},-\frac{x'_1(s)}{|x'(s)|}\Big) \nonumber. 
 \end{split}
 \end{equation}
 
Відомими способами знаходимо невідомі густини з системи \eqref{paramSystem}.

% 4 // Алгоритм %

\section {Алгоритм знаходення розв'язку оберненої крайової задачі}

Знаходження розв'язку задачі \eqref{system}-\eqref{eq} складається з наступного ітераційного процесу:
\begin{itemize}
  \item Для заданого початкової $\Gamma_1$ розв'язуємо пряму задачу для \eqref{system} і знаходимо невідомі густини.
  \item Лінеаризуємо рівняння \eqref{eq} і покращуємо $\Gamma_1$, розв'язуючи лінеаризоване рівняння \eqref{eq} для фіксованих густин, які є відомими з \eqref{system}.
\end{itemize}
Ми припускаємо, що крива $\Gamma_1$ належить класу так званих "зіркових" кривих. Таким чином, параметризація задається як
$x_1(t)=\{r(t)c(t) : t\in[0,2\pi] \}$, де $c(t)=(\cos (t), \sin (t))$ і $\ r : \mathbb{R} \to (0, \infty)$ є $2\pi$-періодичною функцією, яка представляє радіус.

\subsection {Похідна Фреше}

Подамо рівняння \eqref{eq} через оператори. 

 \begin{equation}
 \begin{split}
	&(S_{k}\varphi)(s)=\frac{1}{2\pi}\int_{0}^{2\pi}H_{2k}(s, \sigma)\varphi(\sigma)d\sigma, \\ 
	&(\tilde{S}_{k}\psi)(s)=\frac{1}{2\pi}\int_{0}^{2\pi}\tilde{H}_{2k}(s, \sigma)\psi(\sigma)d\sigma. \nonumber
 \end{split}
 \end{equation}
 
Нехай $r=21$. Тоді \eqref{eq} можна записати так

  \begin{equation}
	S_{1}\varphi_1+\tilde{S}_{1}\psi_1+S_{2}\varphi_2+\tilde{S}_{2}\psi_2+\omega=f \quad \textrm{на } \Gamma_2. \\ 
 \end{equation}
 
  Нехай $q$ радіальна функція, яка задає покращення для $\Gamma_1$. Похідна Фреше для оператора $S$ відносно функціі $\varphi_1$ матиме вигляд
 
\begin{equation}
	(S^{'}[r,\varphi]q)(s)=\frac{1}{8\pi}\int_{0}^{2\pi}q(\sigma)N_{r}(s, \sigma)\varphi(\sigma)d\sigma, 
 \end{equation}
 де $N_{r}(s, \sigma)=c(\sigma)\cdot\nabla_{x_{1}(\sigma)}|x_2(s)-x_1(\sigma)|^2\ln |x_2(s)-x_1(\sigma)|$, $c(\sigma)=(\cos (\sigma), \sin (\sigma))$.
 
 Аналогічно для оператора $\tilde{S}$,
 
\begin{equation}
	(\tilde{S}^{'}[r,\psi]q)(s)=\frac{1}{8\pi}\int_{0}^{2\pi}q(\sigma)\tilde{N}_{r}(s, \sigma)\psi(\sigma)d\sigma,
 \end{equation}
 $\tilde{N}_{r}(s, \sigma)=...$.
\end{document}