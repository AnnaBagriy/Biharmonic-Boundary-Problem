\documentclass[10pt]{beamer}
\usetheme{CambridgeUS}
\usecolortheme{seahorse}
    
\usepackage[utf8]{inputenc}	
\usepackage{mathtools}
\usepackage[english,ngerman,ukrainian]{babel}
\usepackage{amssymb}
\usepackage{graphicx}

\makeatletter

\newcases{rightcases}{\quad}{%
  \hfil$\m@th\displaystyle{##}$}{$\m@th\displaystyle{##}$\hfil}{\lbrace}{.}
\makeatother
 
 
%Information to be included in the title page:
\title[М-д інтегральних р-нь]{Метод інтегральних рівнянь для крайових задач для бігармонійного рівняння}
\author{Багрій Анна}
\institute[ЛНУ]{Львівський національний університет імені Івана Франка}
\date{2019}
 
 
 
\begin{document}
 
\frame{\titlepage}
 
\begin{frame}
\frametitle{Фізичний зміст задачі}
	Розглянемо таке рівняння
	\begin{equation}
		\rho u\textsubscript{tt}(x,t)+D\Delta^{2} u(x,t)=0,
		\quad x\in \Omega\subseteq R^2,t>0,
	\end{equation}
	\begin{center}
		$\Omega$ - деяка пластина,
	\end{center} 
	$$\Delta^{2}=\bigg(\frac{\partial^{2}}{\partial x\textsubscript{1}^2}+\frac{\partial^{2}}{\partial x\textsubscript{2}^2}\bigg)^2 \textrm{- бігармонійний оператор}$$
\end{frame}

\begin{frame}
\frametitle{Постановка задачі}
\begin{equation}
	\label{mainSys}
	\left\{
	\begin{split}
		&\Delta^2 u(x)=0, \quad x\in\Omega, \\
		&u(x)=f_1(x), \quad x\in\partial\Omega_1, \\
		&u(x)=f_2(x), \quad x\in\partial\Omega_2, \\
		&\frac{\partial u(x)}{\partial n}=g_1(x) \quad x\in\partial\Omega_1, \\
		&\frac{\partial u(x)}{\partial n}=g_2(x) \quad x\in\partial\Omega_2. \\
	\end{split}
	\right.
\end{equation}

Знайти таку функцію $u\in C^4( \overline{\Omega})$, що заводовольняє бігармонійне рівняння і крайові умови.
\end{frame}

%Непрямий метод інтегральних рівнянь
\begin{frame}
\frametitle{Непрямий метод інтегральних рівнянь}
Фундаментальний розв'язок $G$ бігармонійного рівняння
\begin{equation}
	\Delta_x^2 G(x, y)=\Delta_y^2 G(x, y)=\delta(x-y), \quad x,\ y \in R^2
\end{equation}
 має вигляд
\begin{equation}
	G(x, y)=\frac{1}{8\pi}|x-y|^2\ln|x-y|.
\end{equation}

\end{frame}


\begin{frame}
\frametitle{Непрямий метод інтегральних рівнянь}
Об'ємні потенціали відповідно простого та подвійного шару мають вигляд
\begin{equation}
\begin{split}
\label{poten}
	V_1(\varphi)(x)=\int_\Omega G(x,y)\varphi(y)d\sigma_y, \\
	V_2(\psi)(x)=\int_\Omega\frac{\partial}{\partial n_y}G(x,y)\psi(y) d\sigma_y,
\end{split}
\end{equation}
$\varphi$ i $\psi$ - густини потенціалів.
Розв'язок з-чі \eqref{mainSys} можна подати як
\begin{gather}
 \label{w}
 	 u(x)=\int_{\partial\Omega}(G(x,y)\varphi(y)+\frac{\partial G(x,y)}{\partial n_y}\psi(y))d\sigma_y, \quad x\in\Omega.
 \end{gather}

\end{frame}

\begin{frame}
\frametitle{Непрямий метод інтегральних рівнянь}
Ядра в потенціалах \eqref{poten} мають вигляд
 \begin{equation}
 	\frac{\partial G(x,y)}{\partial n_x}=\frac{1}{8\pi}n(x)\cdot(x-y)(1+2\ln|x-y|),
 \end{equation}
 \begin{equation}
 	\frac{\partial G(x,y)}{\partial n_y}=-\frac{1}{8\pi}n(y)\cdot(x-y)(1+2\ln|x-y|),
 \end{equation}
 \begin{gather}
 	\frac{\partial^2 G(x,y)}{\partial n_x\partial n_y}=\frac{\partial^2 G(x,y)}{\partial n_y\partial n_x}=-\frac{1}{8\pi}\bigg(2\frac{n(x)\cdot(x-y)n(y)\cdot(x-y)}{|x-y|^2} \\
	+n(x)\cdot n(y)(1+2\ln|x-y|)\bigg) \nonumber.
 \end{gather}
\end{frame}

\begin{frame}
\frametitle{Непрямий метод інтегральних рівнянь}
Теорема. Розв'язок крайової задачі \eqref{mainSys} можна подати у вигляді
\begin{equation}
	 	u(x)=\sum_{k=1}^{2}\int_{\partial\Omega_k}\bigg(G(x,y)\varphi_k(y)+\frac{\partial G(x,y)}{\partial n_y}\psi_k(y)\bigg)d\sigma_y+\omega(x), \quad x\in \Omega,
\end{equation}
$\omega(x) = \alpha_0+\alpha_1x_1+\alpha_2x_2 \ ((\alpha_0,\alpha_1,\alpha_2)\in R^3), \varphi_k,\psi_k\in C(\partial\Omega_k), k=1,2,$
і є єдиним для с-ми
\begin{equation}
	 \left\{
	 	\begin{split}
		\label{system}
	 		&\sum_{k=1}^{2}\int_{\partial\Omega_k}\bigg(G(x,y)\varphi_k(y)+\frac{\partial G(x,y)}{\partial n_y}\psi_k(y)\bigg)d\sigma_y+\omega(x)=f(x), \\
			&\sum_{k=1}^{2}\int_{\partial\Omega_k}\bigg(\frac{\partial G(x,y)}{\partial n_x}\varphi_k(y)+\frac{\partial^2 G(x,y)}{\partial n_y\partial n_x}\psi_k(y)\bigg)d\sigma_y+\frac{\partial\omega(x)}{\partial n}=g(x), \\
		\end{split}
	\right.
	 \end{equation}
	 $$ x\in\partial\Omega_l, \ l=1,2,$$
\end{frame}

\begin{frame}
\frametitle{Непрямий метод інтегральних рівнянь}
разом з рівняннями
\begin{equation}
	 \left\{
	 	\begin{split}
		\label{system}
			&\sum_{k=1}^{2}\int_{\partial\Omega_k}\varphi_k(y)d\sigma_y=A_0, \\
			&\sum_{k=1}^{2}\int_{\partial\Omega_k}(y_1\varphi_k(y)+n_1(y)\psi_k(y))d\sigma_y=A_1, \\
			&\sum_{k=1}^{2}\int_{\partial\Omega_k}(y_2\varphi_k(y)+n_2(y)\psi_k(y))d\sigma_y=A_2
		\end{split}
	\right.
	 \end{equation}
	 для заданих $(A_0,A_1,A_2)\in R^3$.
\end{frame}

\begin{frame}
\frametitle{Параметризація}
$$\partial\Omega_l=\left\{x_l(s)=(x_1{_l}(s),x_2{_l}(s)) \ : \ s\in [0,2\pi]\right\}.$$
 \begin{equation}
 		\left\{
	 	\begin{split}
		\label{paramSystem}
	 		&\frac{1}{2\pi}\sum_{k=1}^{2}\int_{0}^{2\pi}\bigg(H_l{_k}(s, \sigma)\varphi_k(\sigma)+\tilde{H}_l{_k}(s, \sigma)\psi_k(\sigma)\bigg)d\sigma+\omega(x_l(s))=f(x_l(s)), \\
			&\frac{1}{2\pi}\sum_{k=1}^{2}\int_{0}^{2\pi}\bigg(\tilde{\tilde{H}}_l{_k}(s, \sigma)\varphi_k(\sigma)+\hat{H}_l{_k}(s, \sigma)\psi_k(\sigma)\bigg)d\sigma+\frac{\partial\omega(x_l(s))}{\partial n_l}=g(x_l(s)), \\
			&\sum_{k=1}^{2}\int_{0}^{2\pi}\varphi_k(\sigma)d\sigma=A_0, \\
			&\sum_{k=1}^{2}\int_{0}^{2\pi}(x_1{_k}\varphi_k(\sigma)+n_1(x_k(\sigma))\psi_k(\sigma))d\sigma=A_1, \\
			&\sum_{k=1}^{2}\int_{0}^{2\pi}(x_2{_k}\varphi_k(\sigma)+n_2(x_k(\sigma))\psi_k(\sigma))d\sigma=A_2, \\
		\end{split}
		\right.
\end{equation}
$\ l=1,2, \ s\in [0,2\pi]$.
\end{frame}

\begin{frame}
\frametitle{Параметризація}
Тут $ \label{kernels} \varphi_l(s) :=\varphi_k(x_l(s))|x'_l(s)|,  \psi_l(s) :=\psi_k(x_l(s))|x'_l(s)| - \textrm{невідомі густини} $ і ядра мають вигляд
 
 \begin{equation}
 \begin{split}
	&H_l{_k}(s, \sigma) = G(x_l(s),x_k(\sigma)), \quad \tilde{H}_l{_k}(s, \sigma)=\frac{\partial G(x_l(s),x_k(\sigma))}{\partial n_y}, \\
	&\tilde{\tilde{H}}_l{_k}(s, \sigma)=\frac{\partial G(x_l(s),x_k(\sigma))}{\partial n_x}, \quad \hat{H}_l{_k}(s, \sigma)=\frac{\partial^2 G(x_l(s),x_k(\sigma))}{\partial n_y\partial n_x}, \\
	&n(x(s))=\Big(\frac{x'_2(s)}{|x'(s)|},-\frac{x'_1(s)}{|x'(s)|}\Big) - \textrm{зовнішня нормаль.} \nonumber
 \end{split}
 \end{equation}
\end{frame}

\begin{frame}
\frametitle{Параметризація}
При $l=k$ ядра мають логарифімічну особливість. Після її виділення отримаємо такий вигляд для кожного з ядер
$$
H_{ll}(s, \sigma)=H^{(1)}_{ll}(s, \sigma)\ln\bigg(\frac{4}{e}\sin^2\frac{s-\sigma}{2}\bigg)+H^{(2)}_{ll}(s, \sigma),
$$
$$
H^{(1)}_{ll}(s, \sigma)=\frac{1}{8}|x_l(s)-x_l(\sigma)|^2,
$$
$$
H^{(2)}_{ll}(s, \sigma)=\frac{1}{8}|x_l(s)-x_l(\sigma)|, \ s\in [0,2\pi].
$$
Аналогічно для інших ядер.
\end{frame}
 
 \begin{frame}
\frametitle{Чисельне розв'язування}
Застосуємо квадратурні формули 
 \begin{gather}
 	\frac{1}{2\pi}\int_{0}^{2\pi}f(\sigma)d\sigma\approx\frac{1}{2m}\sum_{j=0}^{2m-1}f(s_j), \\
	\frac{1}{2\pi}\int_{0}^{2\pi}f(\sigma)\ln\bigg(\frac{4}{e}\sin^2\frac{s-\sigma}{2}\bigg)d\sigma\approx\frac{1}{2m}\sum_{j=0}^{2m-1}R_j(s)f(s_j) \\
	s_k=kh, \ k=0,...,2m-1, \ h=\frac{\pi}{m}, \nonumber \\
	R_k(s)=-\frac{1}{m}\bigg(\frac{1}{2}+\sum_{j=1}^{m-1}\frac{1}{j}\cos \frac{jk\pi}{m}+ \frac{(-1)^k}{2m}\bigg) - \textrm{ваги} \nonumber ,
 \end{gather}
  до параметризованої системи \eqref{paramSystem}
\end{frame}

\begin{frame}
\frametitle{Чисельне розв'язування}
\begin{equation} \label{lastSys}
  \left\{
  \begin{split}
  	&\sum_{j=0}^{2m-1}\bigg((H^{(1)}_{ll}(s_i, s_j)R_{|i-j|}+\frac{1}{2m}H^{(2)}_{ll}(s_i, s_j))\varphi_{lj}+\frac{1}{2m}H_{l,3-l}(s_i, s_j)\varphi_{3-l,j} \\ 
	&+(\tilde{H}^{(1)}_{ll}(s_i, s_j)R_{|i-j|} +\frac{1}{2m}\tilde{H}^{(2)}_{ll}(s_i, s_j))\psi_{lj}+\frac{1}{2m}\tilde{H}_{l,3-l}(s_i, s_j)\psi_{3-l,j}\bigg) + \omega_{li}=f_{li},\\
	 &\sum_{j=0}^{2m-1}\bigg(\frac{1}{2m}\tilde{\tilde{H}}_{l,3-l}\varphi_{3-lj}+(\tilde{\tilde{H}}^{(1)}_{ll}(s_i, s_j)R_{|i-j|}+\frac{1}{2m}\tilde{\tilde{H}}^{(2)}_{ll}(s_i, s_j))\varphi_{lj} + \frac{1}{2m}\hat{H}_{l,3-l}\psi_{3-lj}\\
	 &+(\hat{H}^{(1)}_{ll}(s_i, s_j)R_{|i-j|}+\frac{1}{2m}\hat{H}^{(2)}_{ll}(s_i, s_j))\psi_{lj}\bigg) +\frac{\partial\omega_{li}}{\partial n_l}=f_{li},\ l=1,2,
\end{split}
\right.
\end{equation}
\end{frame}

\begin{frame}
\frametitle{Чисельне розв'язування}
\begin{equation} \label{lastSys}
  \left\{
  \begin{split}
	 &h\sum_{k=1}^{2}\sum_{j=0}^{2m-1}\varphi_{kj}=A_0, \\
	 &h\sum_{k=1}^{2}\sum_{j=0}^{2m-1}(x_{1k}(s_j)\varphi_{kj}+n_1(x_k(s_j))\psi_{kj})=A_1, \\
	 &h\sum_{k=1}^{2}\sum_{j=0}^{2m-1}(x_{2k}(s_j)\varphi_{kj}+n_2(x_k(s_j))\psi_{kj})=A_2
\end{split}
\right.
\end{equation}
\end{frame}

\begin{frame}
\frametitle{Чисельне розв'язування}
Розв'язок має вигляд
 \begin{equation} \label{solutionEq}
 \begin{split}
 	u(x) \approx\sum_{k=1}^{2}\sum_{j=0}^{2m-1}\bigg(H_k(x,s_j)\varphi_{kj}+\tilde{H}_k(x,s_j)\psi_{kj}\bigg) +a_0+a_1x_1+a_2x_2,\\
 	x\in\Omega, \ \varphi_{kj}\approx\varphi_k(s_j), \ \psi_{kj}\approx\psi_k(s_j), \ j=0,...,2m-1.
 \end{split}
 \end{equation}
\end{frame}
 
 %Приклад 1
\begin{frame}
\frametitle{Чисельні експерименти}
\framesubtitle{Приклад 1}
\begin{equation}
 \begin{split}
 \label{bound}
	&f(x_k)=x_{1k}-2x_{2k}, \ x\in\partial\Omega_k, \\
	&g(x_k)=\frac{f(x_k)}{\partial n_k}, \ x\in\partial\Omega_k, \quad k =1,2. 
 \end{split}
 \end{equation}
 \begin{figure}[h!]
\centering
	\includegraphics[scale=.5]{domain4.pdf}
	\vspace*{-1cm}
\end{figure}
\end{frame}

\begin{frame}
\frametitle{Чисельні експерименти}

\end{frame}

 %Приклад 2
\begin{frame}
\frametitle{Чисельні експерименти}

\end{frame}

\begin{frame}
\frametitle{Чисельні експерименти}

\end{frame}

%Висновки
\begin{frame}
\frametitle{Висновок}

\end{frame}


%Висновки
\begin{frame}
\frametitle{Список літератури}
\begin{thebibliography}{99}
 \addcontentsline{toc}{chapter}{Література}
\bibitem{chen} 
Chen G., Boundary Element Methods with Applications to Nonlinear Problems / Goong Chen, Jianxin Zhou. - Atlantis Press, 2010. - 715 p.

\bibitem{chapko} 
Chapko R. Integral Equations for Biharmonic Data Completion / Roman Chapko, B. Tomas Johansson. - 2019. - 16 p.
 
\bibitem{kress} 
Kress R. Linear Integral Equations / Rainer Kress. - New York : Springer, 1989. - 412 p.

\end{thebibliography}
\end{frame}

\begin{frame}{}
  \centering \huge
  Дякую за увагу
\end{frame}
 
\end{document}