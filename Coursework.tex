\documentclass[12pt]{report}

\usepackage{lingmacros}
\usepackage{tree-dvips}
\usepackage{amsmath}
\usepackage{amsfonts}
\usepackage{hyperref}
\usepackage[english,ukrainian]{babel}
\usepackage{amsthm}
\usepackage{mathtools}

\usepackage[pdftex]{graphicx}
\usepackage{pdfpages}
\usepackage[a4paper,margin=2cm]{geometry}

\usepackage[pdftex]{graphicx}
\usepackage{pdfpages}
\usepackage[a4paper,margin=2cm]{geometry}

%% Useful packages
\usepackage{amsmath}
\usepackage{graphicx}
\usepackage{hyperref}
\usepackage{systeme}
\usepackage{titletoc}
\usepackage{array}
\usepackage{subfig}
\usepackage[font=small,labelfont=bf]{caption}
\usepackage{graphicx}
\usepackage{capt-of}

\usepackage{etoolbox}
\makeatletter
\patchcmd{\thebibliography}{%
  \chapter*{\bibname}\@mkboth{\MakeUppercase\bibname}{\MakeUppercase\bibname}}{%
  \chapter*{Література}}{}{}
\makeatother

\newtheorem{theorem}{Теорема}

\begin{document}
\begin{titlepage}
		\begin{center}
			{\largeЛЬВІВСЬКИЙ НАЦІОНАЛЬНИЙ УНІВЕРСИТЕТ ІМЕНІ ІВАНА ФРАНКА}\\
			{\Large Факультет прикладної математики та інформатики \\
			 Кафедра обчислювальної математики}
		\end{center}
	    \leavevmode \\
	    \leavevmode \\
	    \leavevmode \\
	    \leavevmode \\
		\begin{center}
			{\LARGE  Курсова робота\\}
			 на тему: \\
			\leavevmode \\
		    {\Huge \textbf{Метод інтегральних рівнянь для оберненої крайової задачі для бігармонійного рівняння}}			
		\end{center}
	    \leavevmode \\
	    \leavevmode \\
	    \leavevmode \\
	    \leavevmode \\	
	    \leavevmode \\
	    \leavevmode \\
	    \leavevmode \\
	    \leavevmode \\     
	        \begin{tabular}{p{7cm}p{12cm}}
	    	    \, & {\large Студентки IV курсу групи ПМП-41} \\
	    	    \, & {\large Напряму підготовки ``Прикладна математика''} \\
	    	    \, & {\large Багрій А. Г.} \\
	    	    \, & \, \\
	    	    \, & {\large Керівник:} \\
	    	    \, & {\large проф. Хапко Р. С.} \\
	    	    \, & {\large Національна шкала: \underline{\hspace{3cm}}} \\
	    	    \, & {\large Кількість балів: \underline{\hspace{3cm}}} \\
		    \, & {\large Оцінка ECTS: \underline{\hspace{3cm}}}
	        \end{tabular}
        \leavevmode \\
        \leavevmode \\
        \leavevmode \\
        \leavevmode \\
        \vfill
        \begin{center}
        	{\Large Львів-2020}
        \end{center}
\end{titlepage}

\tableofcontents

% 1 // Формулювання %

\chapter*{Вступ}
\addcontentsline{toc}{chapter}{Вступ}
 
 \quad Обернені задачі, які полягають у реконструкції межі області, є популярним напрямом для дослідження, оскільки такі задачі мають велику кількість застосувань. Реконструкція межі області є нелінійною задачею, що призводить до труднощів як в теоретичних дослідженнях, так і в чисельному розв'язуванні. Найбільш ефективні чисельні методи для даної задачі базуються на інтегральних рівняннях. 
 
 У даній роботі розглянуто реконструкцію внутрішньої межі двозв'язної пластини із заданими крайовими умовами, що задають певний фізичний сенс. Оскільки обернені задачі вимагають розв'язності відповідної прямої задачі, то також показано її коректність і наведено чисельні експеременти, що підтверджують теоретичні обґрунтування.

 \section*{Постановка задачі}
Нехай $\Omega\subset \mathbb{R}^2$ деяка двоз'язна область з межею $\Gamma=\Gamma_1\cup\Gamma_2$, де $\Gamma_1$ є внутрішньою межею і $\Gamma_2$ - зовнішньою. Розглянемо

\begin{equation}
	\left\{
	\label{directProblem}
	\begin{split}
		&\Delta^2 u(x)=0, \quad x\in\Omega, \\
		&u(x)=\frac{\partial u(x)}{\partial n}=0, \quad x\in\Gamma_1, \\
		&\frac{\partial u(x)}{\partial n}=g(x), \quad x\in\Gamma_2, \\
		&Mu(x)=q(x), \quad x\in\Gamma_2,
	\end{split}
	\right.
\end{equation}

\begin{equation}
	\label{dataEq}
	u(x)=f(x), \quad x\in\Gamma_2,
\end{equation}
де 
$$\Delta^{2}u(x)=\frac{\partial^{4}}{\partial x_1^4}u(x)+2\frac{\partial^{4}}{\partial x_1^2\partial x_2^2}u(x)+\frac{\partial^{4}}{\partial x_2^4}u(x) \textrm{ - бігармонійний оператор,}
$$
 $$u(x)=u(x_1,x_2) \in C^2(\Omega)\cap C^1(\overline{\Omega}) \textrm{ - функція, що задовольняє рівняння і крайові умови}, $$
 $$n=(n_1, n_2) \textrm{ - зовнішня нормаль до }\Gamma,$$ 
 $$Mu=\nu\Delta u+(1-\nu)(u_{x_1 x_1}n_1^2+2u_{x_1 x_2}n_1 n_2+u_{x_2 x_2}n_2^2) \textrm{ - згинальний момент пластини } \Omega,
 $$ 
 $\nu\in (0, 1)$ - коефіцієнт Пуассона і $g(x), \ q(x),\  f(x)$ - деякі задані функції, тотожньо відмінні від нуля.
 
Обернена задача \eqref{directProblem}-\eqref{dataEq} полягає у знаходженні невідомої межі $\Gamma_1$ для заданих функцій $g(x), q(x), f(x)$. Дана задача є нелінійною, оскільки її розв'язок нелінійно залежить від невідомої межі $\Gamma_1$, і також некоректною в сенсі відсутності стійкості за вхідними данними.

Перш за все, необхідно розглянути питання існування та єдиності розв'язку оберненої задачі \eqref{directProblem}-\eqref{dataEq}.

\begin{theorem}[Існування і єдиність розв'язку оберненої задачі]

Нехай $\tilde{\Gamma}_1$, $\Gamma_1$ - замкнені криві, що містяться всередині $\Gamma_2$, $\tilde{u}$, $u$ - розв'язки задачі $\eqref{directProblem}$ для $\tilde{\Gamma}_1$ і $\Gamma_1$ відповідно. Нехай $g\neq 0, q\neq 0$ i $u = \tilde{u}$ на відкритій підмножині $\Gamma_2$. Тоді $\tilde{\Gamma}_1=\Gamma_1$.

\end{theorem}

\begin{proof}

Доведемо від протилежного. Нехай $\tilde{\Gamma}_1\neq\Gamma_1$.
Нехай $\Omega_2$ - область, обмежена $\Gamma_2$, $\Omega_1$, $\tilde{\Omega}_1$ - області обмежені $\Gamma_1$, $\tilde{\Gamma}_1$ відповідно. 
\begin{equation}
W:=\Omega_2\setminus(\Omega_1\cup\tilde{\Omega}_1), \quad \Gamma_2\subset\partial W. \nonumber
\end{equation}
За теоремою Гольмгрена \cite{Holmgren} маємо $u = \tilde{u}$ в $W$. Без втрати загальності приспустимо, що $W^{*}:=(\Omega_2\setminus \overline{W})\setminus\Omega_1$ непорожня множина. Тоді $u$ є визначена в $W^{*}$ як розв'язок задачі \eqref{directProblem} для $\Gamma_1$. Вона є неперервною в $\overline{W}^{*}$ і задовольняє граничні умови на $\partial W^{*}$. Ця гранична умова випливає з того, що кожна точка з $W^{*}$ належить або $\Gamma_1$, або $\partial W\cap\tilde{\Gamma}_1$. Для $x\in\Gamma_1$ маємо $u(x)=0$ як наслідок граничних умов для $u$, для $x\in\tilde{\Gamma}_1$ маємо $u(x)=\tilde{u}(x)$ і тому $u(x)=0$ як наслідок граничних умов для $\tilde{u}$. Тоді за принципом максимуму $u=0$ в $W^{*}$ і тому $u=0$ в $D$. Це суперечить тому, що $g\neq 0, q\neq 0$.

\end{proof}


% 2 // Теореми %
\setcounter{secnumdepth}{1}
\chapter{Загальні положення}
\label{chapter1}

\section{Зведення до ІР}

Фундаментальний розв'язок бігармонійного рівняння в $\mathbb{R}^2$ має вигляд
 \begin{equation}
 	G(x, y)=\frac{1}{8\pi}|x-y|^2\ln|x-y|, \quad x,\ y \in \mathbb{R}^2.
 \end{equation}

Розглянемо потенціал простого шару для бігармонійного рівняння \cite{chapko} з густинами $\varphi$, $\psi$, що визначені на $\Gamma$:

\begin{equation}
	 	u(x)=\int_{\Gamma}\bigg(G(x,y)\varphi(y)+\frac{\partial G(x,y)}{\partial n_y}\psi(y)\bigg)d\sigma_y, \quad x\in \Omega. \nonumber
\end{equation}

\begin{theorem}[Існування і єдиність розв'язку прямої задачі]
Існує єдиний розв'язок прямої крайової задачі \eqref{directProblem}, який можна подати у вигляді
	 \begin{equation}
	 	u(x)=\sum_{k=1}^{2}\int_{\Gamma_k}\bigg(G(x,y)\varphi_k(y)+\frac{\partial G(x,y)}{\partial n_y}\psi_k(y)\bigg)d\sigma_y+\omega(x), \quad x\in \Omega,
	 \end{equation}
	 де $\omega(x) = \alpha_0+\alpha_1x_1+\alpha_2x_2 \ ((\alpha_0,\alpha_1,\alpha_2)\in R^3), \varphi_k,\psi_k\in C(\Gamma_k), k=1,2$, і невідомі густини визначаються з системи інтегральних рівнянь
	 \begin{equation}
	 \left\{
	 	\begin{split}
		\label{system}
	 		&\sum_{k=1}^{2}\int_{\Gamma_k}\bigg(G(x,y)\varphi_k(y)+\frac{\partial G(x,y)}{\partial n_y}\psi_k(y)\bigg)d\sigma_y+\omega(x)=0, \ x\in\Gamma_1, \\
			&\sum_{k=1}^{2}\int_{\Gamma_k}\bigg(\frac{\partial G(x,y)}{\partial n_x}\varphi_k(y)+\frac{\partial^2 G(x,y)}{\partial n_y\partial n_x}\psi_k(y)\bigg)d\sigma_y+\frac{\partial\omega(x)}{\partial n}=0, \ x\in\Gamma_1, \\
			&\sum_{k=1}^{2}\int_{\Gamma_k}\bigg(\frac{\partial G(x,y)}{\partial n_x}\varphi_k(y)+\frac{\partial^2 G(x,y)}{\partial n_y\partial n_x}\psi_k(y)\bigg)d\sigma_y+\frac{\partial\omega(x)}{\partial n}=g(x), \ x\in\Gamma_2, \\
			&\sum_{k=1}^{2}\int_{\Gamma_k}\bigg(M_x G(x,y)\varphi_k(y)+\frac{\partial M_x G(x,y)}{\partial n_y}\psi_k(y)\bigg)d\sigma_y+M_x\omega(x)=q(x), \ x\in\Gamma_2, \\
			&\sum_{k=1}^{2}\int_{\Gamma_k}\varphi_k(y)d\sigma_y=A_0, \\
			&\sum_{k=1}^{2}\int_{\Gamma_k}(y_1\varphi_k(y)+n_1(y)\psi_k(y))d\sigma_y=A_1, \\
			&\sum_{k=1}^{2}\int_{\Gamma_k}(y_2\varphi_k(y)+n_2(y)\psi_k(y))d\sigma_y=A_2,
		\end{split}
	\right.
	 \end{equation}
	для заданих $(A_0,A_1,A_2)\in R^3$.
 \end{theorem}
 
 Константи $A_0, A_1, A_2$ вибираються довільним чином, однак є некоректним брати за значення нулі.
 
 З умови \eqref{dataEq} маємо наступне інтегральне рівняння
 \begin{equation}
 \label{dataEq1}
	\sum_{k=1}^{2}\int_{\Gamma_k}\bigg(G(x,y)\varphi_k(y)+\frac{\partial G(x,y)}{\partial n_y}\psi_k(y)\bigg)d\sigma_y+\omega(x)=f(x), \ x\in\Gamma_2.
 \end{equation}
 

В системі \eqref{system} ядра мають вигляд
 \begin{equation*}
 	\frac{\partial G(x,y)}{\partial n_y}=-\frac{1}{8\pi}n(y)\cdot(x-y)(1+2\ln|x-y|),
 \end{equation*}
\begin{equation*}
 	\frac{\partial G(x,y)}{\partial n_x}=\frac{1}{8\pi}n(x)\cdot(x-y)(1+2\ln|x-y|),
 \end{equation*}
 \begin{gather*}
 	\frac{\partial^2 G(x,y)}{\partial n_x\partial n_y}=\frac{\partial^2 G(x,y)}{\partial n_y\partial n_x}=-\frac{1}{8\pi}\bigg(2\frac{n(x)\cdot(x-y)n(y)\cdot(x-y)}{|x-y|^2} \\
	+n(x)\cdot n(y)(1+2\ln|x-y|)\bigg),
 \end{gather*}
 \begin{gather*}
 	M_x G(x,y)=\frac{1+3\nu}{8\pi}+\frac{(1-\nu)(n(x)\cdot(x-y))^2}{4\pi|x-y|^2}+\frac{(1+\nu)\ln|x-y|^2}{8\pi},
 \end{gather*}
 \begin{gather*}
 	\frac{\partial M_x G(x,y)}{\partial n_y}=\frac{1-\nu}{2\pi}\Big(\frac{(n(x)\cdot(x-y))^2n(y)\cdot(x-y)}{|x-y|^4}-\frac{n(x)\cdot n(y)n(x)\cdot(x-y)}{|x-y|^2} \Big) \\
	-\frac{(1+\nu)n(y)\cdot(x-y)}{4\pi|x-y|^2}.
 \end{gather*}

\begin{theorem}
Обернена крайова задача \eqref{directProblem}-\eqref{dataEq} еквівалентна системі інтегральних рівнянь \eqref{system}-\eqref{dataEq1}.
\end{theorem}

 % 
\section{Алгоритм знаходення розв'язку оберненої крайової задачі}

Знаходження розв'язку задачі \eqref{system}-\eqref{dataEq1} складається з наступного ітераційного процесу:
\begin{itemize}
  \item Для заданого початкового наближення $\Gamma_1$ розв'язуємо пряму задачу для \eqref{system} і знаходимо невідомі густини.
  \item Лінеаризуємо рівняння \eqref{dataEq1} і покращуємо $\Gamma_1$, розв'язуючи лінеаризоване рівняння \eqref{dataEq1} для фіксованих густин, які є відомими з \eqref{system}.
  \item Умовою зупинки може бути $||q||_2  <\epsilon$, де $q$ - функція, що задає покращення  $\Gamma_1$, $\epsilon$ - задана точність.
\end{itemize}


% 3 // ПРЯМА ЗАДАЧА %
\chapter{Чисельне розв'язування коректної системи ІР}

% 3.1 // Параметризація %
\section{Параметризація}


Припустимо, що криві $\Gamma_1$ i $\Gamma_2$ достатньо гладкі і їх можна подати у параметричному заданні 
 
 \begin{equation}
 	\Gamma_l=\left\{x_l(s)=(x_1{_l}(s),x_2{_l}(s)) \ : \ s\in [0,2\pi]\right\},
  \end{equation}
 де $x_l \ (l=1,2)$ - аналітична, $2\pi$-періодична функція, $|x'(s)|>0.$
 Тоді систему \eqref{system} можна подати у вигляді
 
 \begin{equation}
 		\left\{
	 	\begin{split}
		\label{paramSystem}
	 		&\frac{1}{2\pi}\sum_{k=1}^{2}\int_{0}^{2\pi}\bigg(H_1{_k}(s, \sigma)\varphi_k(\sigma)+\tilde{H}_1{_k}(s, \sigma)\psi_k(\sigma)\bigg)d\sigma+\omega(x_1(s))=0,\\
			&\frac{1}{2\pi}\sum_{k=1}^{2}\int_{0}^{2\pi}\bigg(\tilde{\tilde{H}}_1{_k}(s, \sigma)\varphi_k(\sigma)+\hat{H}_1{_k}(s, \sigma)\psi_k(\sigma)\bigg)d\sigma+\frac{\partial\omega(x_1(s))}{\partial n_1}=0, \\
			&\frac{1}{2\pi}\sum_{k=1}^{2}\int_{0}^{2\pi}\bigg(\tilde{\tilde{H}}_2{_k}(s, \sigma)\varphi_k(\sigma)+\hat{H}_2{_k}(s, \sigma)\psi_k(\sigma)\bigg)d\sigma+\frac{\partial\omega(x_2(s))}{\partial n_2}=g(x_2(s)), \\
			&\frac{1}{2\pi}\sum_{k=1}^{2}\int_{0}^{2\pi}\bigg(\hat{\hat{H}}_2{_k}(s, \sigma)\varphi_k(\sigma)+\bar{H}_2{_k}(s, \sigma)\psi_k(\sigma)\bigg)d\sigma+M_x\omega(x_2(s))=q(x_2(s)),\\
			&\sum_{k=1}^{2}\int_{0}^{2\pi}\varphi_k(\sigma)d\sigma=A_0, \\
			&\sum_{k=1}^{2}\int_{0}^{2\pi}(x_1{_k}\varphi_k(\sigma)+n_1(x_k(\sigma))\psi_k(\sigma))d\sigma=A_1, \\
			&\sum_{k=1}^{2}\int_{0}^{2\pi}(x_2{_k}\varphi_k(\sigma)+n_2(x_k(\sigma))\psi_k(\sigma))d\sigma=A_2, \\
		\end{split}
		\right.
\end{equation}

Тут $ \label{kernels} \varphi_l(s) :=\varphi_k(x_l(s))|x'_l(s)|, \ \psi_l(s) :=\psi_k(x_l(s))|x'_l(s)| \ - \ \textrm{невідомі густини} $ і ядра мають вигляд
 
 \begin{equation}
 \begin{split}
	&H_l{_k}(s, \sigma) = G(x_l(s),x_k(\sigma)), \quad \tilde{H}_l{_k}(s, \sigma)=\frac{\partial G(x_l(s),x_k(\sigma))}{\partial n_y}, \\
	&\tilde{\tilde{H}}_l{_k}(s, \sigma)=\frac{\partial G(x_l(s),x_k(\sigma))}{\partial n_x}, \quad \hat{H}_l{_k}(s, \sigma)=\frac{\partial^2 G(x_l(s),x_k(\sigma))}{\partial n_y\partial n_x}, \\
	&\hat{\hat{H}}_l{_k}(s, \sigma) = M_xG(x_l(s),x_k(\sigma)), \quad \bar{H}_l{_k}(s, \sigma)=\frac{\partial M_xG(x_l(s),x_k(\sigma))}{\partial n_y}, \\
	&n(x(s))=\Big(\frac{x'_2(s)}{|x'(s)|},-\frac{x'_1(s)}{|x'(s)|}\Big) \nonumber. 
 \end{split}
 \end{equation}
 
  Дані ядра є неперервними в області $\bar{\Omega}$. Але коли точка інтегрування співпадає з точкою спостереження на $\Gamma_l \ (l=1,2)$ під час диференціювання маємо логарифмічну особливість. Для чисельного розв'язування доцільно виділити цю особливість, виконавши певні перетворення. Отримуємо наступне подання ядер 
  
 \begin{gather}
 	H_l{_l}(s, \sigma)=H^{(1)}_{ll}(s, \sigma)\ln\bigg(\frac{4}{e}\sin^2\frac{s-\sigma}{2}\bigg)+H^{(2)}_{ll}(s, \sigma), \\
	\tilde{H}_{ll}(s, \sigma)=\tilde{H}^{(1)}_{ll}(s, \sigma)\ln\bigg(\frac{4}{e}\sin^2\frac{s-\sigma}{2}\bigg)+\tilde{H}^{(2)}_{ll}(s, \sigma), \\
	\tilde{\tilde{H}}_l{_l}(s, \sigma)=\tilde{\tilde{H}}^{(1)}_{ll}(s, \sigma)\ln\bigg(\frac{4}{e}\sin^2\frac{s-\sigma}{2}\bigg)+\tilde{\tilde{H}}^{(2)}_{ll}(s, \sigma), \\
	\hat{H}_l{_l}(s, \sigma)=\hat{H}^{(1)}_{ll}(s, \sigma)\ln\bigg(\frac{4}{e}\sin^2\frac{s-\sigma}{2}\bigg)+\hat{H}^{(2)}_{ll}(s, \sigma), \\ 
	\hat{\hat{H}}_l{_l}(s, \sigma)=\hat{\hat{H}}^{(1)}_{ll}(s, \sigma)\ln\bigg(\frac{4}{e}\sin^2\frac{s-\sigma}{2}\bigg)+\hat{\hat{H}}^{(2)}_{ll}(s, \sigma).
 \end{gather}
 де 
\begin{gather*}
	H^{(1)}_{ll}(s, \sigma)=\frac{1}{8}|x_l(s)-x_l(\sigma)|^2, \quad H^{(2)}_{ll}(s, \sigma)=\frac{1}{8}|x_l(s)-x_l(\sigma)|^2\ln\Big(\frac{e|x_l(s)-x_l(\sigma)|^2}{4\sin^2\frac{s-\sigma}{2}}\Big), \\
	\tilde{H}^{(1)}_{ll}(s, \sigma)=-\frac{1}{4}n(x_l(\sigma))\cdot(x_l(s)-x_l(\sigma)), \\
	 \tilde{H}^{(2)}_{ll}(s, \sigma)= \frac{1}{4}n(x_l(\sigma))\cdot(x_l(s)-x_l(\sigma))\bigg(\ln\Big(\frac{4\sin^2\frac{s-\sigma}{2}}{e|x_l(s)-x_l(\sigma)|^2}\Big)-1\bigg),\\
	\tilde{\tilde{H}}^{(1)}_{ll}(s, \sigma)=\frac{1}{4}n(x_l(s))\cdot(x_l(s)-x_l(\sigma)), \\
	 \tilde{\tilde{H}}^{(2)}_{ll}(s, \sigma)= \frac{1}{4}n(x_l(s))\cdot(x_l(s)-x_l(\sigma))\bigg(\ln\Big(\frac{e|x_l(s)-x_l(\sigma)|^2}{4\sin^2\frac{s-\sigma}{2}}\Big)+1\bigg),\\
	\hat{H}^{(1)}_{ll}(s, \sigma)= -\frac{1}{4}n(x_l(s))\cdot n(x_l(\sigma)), \\
	 \hat{H}^{(2)}_{ll}(s, \sigma)=\frac{1}{4}n(x_l(s))\cdot n(x_l(\sigma))\cdot\bigg(\ln\Big(\frac{4\sin^2\frac{s-\sigma}{2}}{e|x_l(s)-x_l(\sigma)|^2}\Big)-2\frac{(x_l(s)-x_l(\sigma))^2}{|x_l(s)-x_l(\sigma)|^2} -1\bigg), \\
	 \hat{\hat{H}}^{(1)}_{ll}(s, \sigma)= \frac{1+\nu}{4}, \\
	 \hat{\hat{H}}^{(2)}_{ll}(s, \sigma) = \frac{1+3\nu}{8\pi}+\frac{(1-\nu)(n(x_l(s))\cdot(x_l(s)-x_l(\sigma)))^2}{4\pi|x_l(s)-x_l(\sigma)|^2}+\frac{(1+\nu)\ln|x_l(s)-x_l(\sigma)|^2}{8\pi}\\
	-\frac{1+\nu}{4}\ln\bigg(\frac{4}{e}\sin^2\frac{s-\sigma}{2}\bigg).
 \end{gather*}
 
 При $s=\sigma$ всі ядра дорівнюють нулю, окрім 
 
 \begin{gather*}
	\hat{\hat{H}}^{(2)}_{ll}(s, s) = \frac{1+3\nu}{4}+\frac{1+\nu}{4}\ln(e|x_l^{'}(s)|^2), \\
	\bar{H}_{ll}(s,s)=\frac{1-3\nu}{4}\frac{n(x_l(s))\cdot x_l^{''}(s)}{|x_l^{'}(s)|^2}.
 \end{gather*}
 
 Параметризований розв'язок прямої задачі \eqref{paramSystem} має вигляд
 
  \begin{equation}
 \begin{split}
	\tilde{u}(x)=\sum_{k=1}^{2}\int_{\Gamma_k}\bigg(H_{k}(x, \sigma)\varphi_k(\sigma)+\tilde{H}_{k}(x, \sigma)\psi_k(\sigma)\bigg)d\sigma_y+\omega(x), \ x=(x_1,x_2)\in\Omega,\\
\end{split}
 \end{equation}
 де
 
 \begin{equation*}
 \begin{split}
	&H_{l}(x, \sigma) = \frac{1}{8\pi}|x-x_l(\sigma)|^2\ln|x-x_l(\sigma)|,\\
	&\tilde{H}_{l}(x, \sigma) = \frac{1}{8\pi}n(x_l(\sigma))\cdot(x-x_l(\sigma))(1+2\ln|x-x_l(\sigma)|), \ l=1, 2.
\end{split}
 \end{equation*}
 
 
% 3.2 // Чисельне розв'язування %
\section{Метод Нистрьома}

Для чисельного розв'язування отриманої системи інтегральних рівнянь \eqref{paramSystem} використуємо метод Нистрьома, який полягає у заміні інтегралу вiдповiдною квадратурною формулою з певними ваговими функцiями \cite{kress}. Виберемо тригонометричні квадратури та рівновіддалений поділ,

\begin{gather}
 	\frac{1}{2\pi}\int_{0}^{2\pi}f(\sigma)d\sigma\approx\frac{1}{2m}\sum_{j=0}^{2m-1}f(s_j), \\
	\frac{1}{2\pi}\int_{0}^{2\pi}f(\sigma)\ln\bigg(\frac{4}{e}\sin^2\frac{s-\sigma}{2}\bigg)d\sigma\approx\frac{1}{2m}\sum_{j=0}^{2m-1}R_j(s)f(s_j)
 \end{gather}
 з вузлами
 \begin{gather}
 	s_k=kh, \ k=0,...,2m-1, \ h=\frac{\pi}{m},
  \end{gather}
  і ваговими функціями
  \begin{gather}
 	R_k(s)=-\frac{1}{m}\bigg(\frac{1}{2}+\sum_{j=1}^{m-1}\frac{1}{j}\cos \frac{jk\pi}{m}+ \frac{(-1)^k}{2m}\bigg).
  \end{gather}

Вибір даних квадратур дає експоненційну збіжність, якщо підінтегральна функція $f$ є аналітичною \cite{kress}.

Застосувавши даний метод до параметризованої системи інтегральних рівнянь \eqref{paramSystem} із врахованими логарифмічними особливостями ядер, отримаємо повністю дискретизовану систему лінійних рівнянь

  \begin{equation} 
  \label{discreteSystem}
  \left\{
  \begin{split}
  	&\sum_{j=0}^{2m-1}\bigg((H^{(1)}_{11}(s_i, s_j)R_{|i-j|}+\frac{1}{2m}H^{(2)}_{11}(s_i, s_j))\varphi_{1j}+\frac{1}{2m}H_{12}(s_i, s_j)\varphi_{2j} +(\tilde{H}^{(1)}_{11}(s_i, s_j)R_{|i-j|} \\
	&+\frac{1}{2m}\tilde{H}^{(2)}_{11}(s_i, s_j))\psi_{1j}+\frac{1}{2m}\tilde{H}_{12}(s_i, s_j)\psi_{2j}\bigg) + \omega_{1i}=0, \\
	%
	 &\sum_{j=0}^{2m-1}\bigg((\tilde{\tilde{H}}^{(1)}_{11}(s_i, s_j)R_{|i-j|}+\frac{1}{2m}\tilde{\tilde{H}}^{(2)}_{11}(s_i, s_j))\varphi_{1j} +\frac{1}{2m}\tilde{\tilde{H}}_{12}\varphi_{2j}+(\hat{H}^{(1)}_{11}(s_i, s_j)R_{|i-j|}\\
	 &+\frac{1}{2m}\hat{H}^{(2)}_{11}(s_i, s_j))\psi_{1j}+\frac{1}{2m}\hat{H}_{12}\psi_{2j}\bigg) +\frac{\partial\omega_{1i}}{\partial n_1}=0,\\
	 %
	 &\sum_{j=0}^{2m-1}\bigg(\frac{1}{2m}\tilde{\tilde{H}}_{21}\varphi_{1j}+(\tilde{\tilde{H}}^{(1)}_{22}(s_i, s_j)R_{|i-j|}+\frac{1}{2m}\tilde{\tilde{H}}^{(2)}_{22}(s_i, s_j))\varphi_{2j} + \frac{1}{2m}\hat{H}_{21}(s_i, s_j)\psi_{1j}\\
	 &+(\hat{H}^{(1)}_{22}(s_i, s_j)R_{|i-j|}+\frac{1}{2m}\hat{H}^{(2)}_{22}(s_i, s_j))\psi_{2j}\bigg) +\frac{\partial\omega_{2i}}{\partial n_2}=g_{2i},\\
	 %
	 &\sum_{j=0}^{2m-1}\bigg(\frac{1}{2m}\hat{\hat{H}}_{21}(s_i, s_j)\varphi_{1j}+(\hat{\hat{H}}^{(1)}_{22}(s_i, s_j)R_{|i-j|}+\frac{1}{2m}\hat{\hat{H}}^{(2)}_{22}(s_i, s_j))\varphi_{2j} + \frac{1}{2m}(\bar{H}_{21}(s_i, s_j)\psi_{1j}\\
	 &+\bar{H}_{22}(s_i, s_j)\psi_{2j}) +M\omega_{2i}=q_{2i},\\
	 %
	 & \\
	 &h\sum_{k=1}^{2}\sum_{j=0}^{2m-1}\varphi_{kj}=A_0, \\
	 &h\sum_{k=1}^{2}\sum_{j=0}^{2m-1}(x_{1k}(s_j)\varphi_{kj}+n_1(x_k(s_j))\psi_{kj})=A_1, \\
	 &h\sum_{k=1}^{2}\sum_{j=0}^{2m-1}(x_{2k}(s_j)\varphi_{kj}+n_2(x_k(s_j))\psi_{kj})=A_2
\end{split}
\right.
\end{equation}

для $i=0,...,2m-1$. Тут $\ g_{2i}=g(x_2(s_i)),\ q_{2i}=q(x_2(s_i)), \ \omega_{1i}=\omega(x_1(s_i)), \ \frac{\partial\omega_{li}}{\partial n_l}=\frac{\partial\omega(x_l(s_i))}{\partial n_l} \ (l=1,2), \ M\omega_{2i}=\omega(x_2(s_i)), \ R_j=R(s_j)$.
Розв'язавши систему \eqref{discreteSystem}, отримаємо шукані коефіцієнти $(a_0,a_1,a_2)\in R^3$ і значення густин потенціалів на вибраному розбитті $\varphi_{kj}\approx\varphi_k(s_j), \ \psi_{kj}\approx\psi_k(s_j), \ k=1,2, \ j=0,...,2m-1.$


% 5 // Алгоритм %

\chapter{Чисельне розв'язування некоректного ІР}

Ми припускаємо, що крива $\Gamma_1$ належить класу так званих "зіркових" кривих. Таким чином, параметризація задається як
$$x_1(t)=\{r(t)c(t) : t\in[0,2\pi] \},$$
де $c(t)=(\cos (t), \sin (t))$ і $\ r : \mathbb{R} \to (0, \infty)$ є $2\pi$-періодичною функцією, яка задає радіус від початку координат.

Тоді рівняння \eqref{dataEq1} можна записати як

 \begin{equation}
 \label{eq}
	 \frac{1}{2\pi}\sum_{k=1}^{2}\int_{0}^{2\pi}\bigg(H_2{_k}(s, \sigma)\varphi_k(\sigma)+\tilde{H}_2{_k}(s, \sigma)\psi_k(\sigma)\bigg)d\sigma+\omega(x_2(s))=f(x_2(s)), 
\end{equation}

Подамо рівняння \eqref{eq} через оператори.

 \begin{equation}
 \begin{split}
	&(S_{k}\varphi)(s):=\frac{1}{2\pi}\int_{0}^{2\pi}H_{2k}(s, \sigma)\varphi(\sigma)d\sigma, \\ 
	&(\tilde{S}_{k}\psi)(s):=\frac{1}{2\pi}\int_{0}^{2\pi}\tilde{H}_{2k}(s, \sigma)\psi(\sigma)d\sigma. \nonumber
 \end{split}
 \end{equation}
 
Нехай $r=21$. Тоді \eqref{eq} можна записати так

  \begin{equation}
	S_{1}\varphi_1+\tilde{S}_{1}\psi_1+S_{2}\varphi_2+\tilde{S}_{2}\psi_2+\omega=f \quad \textrm{на } \Gamma_2. \\ 
 \end{equation}
 
 Для лінеаризації необхідно обрахувати похідні Фреше. Після застосування метода Ньютона лінеаризоване рівняння матиме вигляд
 \begin{equation}
  \label{lineriazed}
   \begin{multlined}
	(S_1^{'}[r,\varphi]q)(s) + (\tilde{S}_1^{'}[r,\psi]q)(s)=\\f(s)-(S_{1}\varphi_1)(s)-(\tilde{S}_{1}\psi_1)(s)-(S_{2}\varphi_2)(s)-(\tilde{S}_{2}\psi_2)(s)-\omega(s),
 \end{multlined}
 \end{equation}
 де $q$ - радіальна функція, що задає покращення для $\Gamma_1$, $(S_1^{'}[r,\varphi]q)(s)$, $(\tilde{S}_1^{'}[r,\psi]q)(s)$ - похідні Фреше операторів $S_1$ i $\tilde{S}_1$ відповідно, і мають наступний вигляд
 
 \begin{equation}
	(S_1^{'}[r,\varphi]q)(s)=\frac{1}{8\pi}\int_{0}^{2\pi}q(\sigma)N_{r}(s, \sigma)\varphi(\sigma)d\sigma, 
 \end{equation}
 де $N_{r}(s, \sigma)=c(\sigma)\cdot\nabla_{x_{1}(\sigma)}|x_2(s)-x_1(\sigma)|^2\ln |x_2(s)-x_1(\sigma)|=c(\sigma)\cdot(x_2(s)-x_1(\sigma))(2\ln |x_2(s)-x_1(\sigma)| + 1)$,
 
 \begin{equation}
	(\tilde{S}_1^{'}[r,\psi]q)(s)=-\frac{1}{8\pi}\int_{0}^{2\pi}q(\sigma)\tilde{N}_{r}(s, \sigma)\psi(\sigma)d\sigma,
 \end{equation}
 $\tilde{N}_{r}(s, \sigma)=c(\sigma)\cdot \nabla_{x_{1}(\sigma)} n(x_1(\sigma))\cdot(x_2(s)-x_1(\sigma))(1+2\ln|x_2(s)-x_1(\sigma)|)=c(\sigma)\cdot n(x_1(\sigma))(-1-\ln |x_2(s)-x_1(\sigma)|+(x_2(s)-x_1(\sigma))(1 - \frac{2(x_2(s)-x_1(\sigma))}{|x_2(s)-x_1(\sigma)|^2} -\frac{x^{'}_1(\sigma)x^{''}_1(\sigma) (1+2\ln |x_2(s)-x_1(\sigma)|)}{|x^{'}_1(\sigma)|^2})$.
 
  Рівняння \eqref{lineriazed} розв'яжемо методом колокацій, апроксимуючи $q$ у вигляді
 \begin{equation}
q_n(s)=\sum_{i=0}^{2n}q_{ni}l_i(s), \quad n\in N, \ n>m, \nonumber
 \end{equation}
 
 де $l_i(s)=\cos is$, коли $i=0,...,n$ i $l_i(s)=\sin(n-i)s$ для $i=n+1,...,2n$.
 
 Тоді необхідно розв'язати таку систему лінійних рівнянь
 \begin{equation}
 \label{illSys}
\sum_{j=0}^{2n}q_{nj}A_{ij}=b_i, \quad i=0,...,2m-1,
 \end{equation} 
 
 де
 \begin{equation}
A_{ij}=\frac{1}{8m}\sum_{k=0}^{2m-1}\Big(l_{j}(s_k)N_r(s_i,s_k)\varphi_{1m}(s_k)+l_{j}(s_k)\tilde{N}_r(s_i,s_k)\psi_{1m}(s_k)\Big) \nonumber
 \end{equation} 
 
 і 
 \begin{equation}
 \begin{multlined}
 b_i=f(s_i)-w(s_i)-\sum_{k=0}^{2m-1}\Big(\frac{1}{2m}H_{21}(s_i,s_k)\varphi_{1n}(s_k)+\frac{1}{2m}\tilde{H}_{21}(s_i,s_k)\psi_{1n}(s_k)+\\\big(R_j(s_i)H_{22}^{(1)}(s_i,s_k)+\frac{1}{2m}H_{22}^{(2)}(s_i,s_k)\big)\varphi_{2n}(s_k)+\big(R_j(s_i)\tilde{H}_{22}^{(1)}(s_i,s_k)+\frac{1}{2m}\tilde{H}_{22}^{(2)}(s_i,s_k)\big)\psi_{2n}(s_k)\Big). \nonumber
 \end{multlined}
 \end{equation} 
 
 Оскільки система \eqref{illSys} є погано обумовленою і перевизначеною, то для знаходження її розв'язку застосуємо метод найменших квадратів і регуляризацію Тихонова з параметром регуляризації $\lambda$. Тоді розв'язок матиме такий вигляд
 
 \begin{equation}
\tilde{q}=(A^T A+\lambda I)^{-1}A^T b, \quad \tilde{q}=(q_{n,1}, ..., q_{n,2n})^T.
 \end{equation} 
 
 Нове наближення радіальної функції $r$ обчислюється як $r=r+q_n$.
 \\
 
 Отже, можна сформулювати наступний ітераційний процес.
\begin{enumerate}
  \item Вибрати початкове наближення для $r$.
  \item Сформувати і розв'язати дискретизовану систему лінійних рівнянь для знаходження невідомих густин $\varphi_k,\psi_k, \ k=1,2$, і констант $a_0,a_1, a_2$.
  \item Для фіксованих $r, a_0,a_1, a_2, \varphi_k,\psi_k, \ k=1,2$ розв'язати лінеаризоване рівняння \eqref{lineriazed} відносно функції $q$, що задає покращення для $\Gamma_1$.
  \item Обрахувати нове наближення для радіальної функції $r=r+q_n$.
  \item Якщо $||q||_2<\epsilon$, то наближення до $\Gamma_1$ знайдено. Інакше перейти до кроку 2.
\end{enumerate}

\chapter{Чисельні експерименти}

Розглянемо приклади чисельного наближення розв'язку прямої крайової задачі \eqref{directProblem} для різних областей, параметрів $A_0,A_1,A_2,\nu$ та заданих функцій $f, q$.

\subsection{Приклад 1}
Розглянемо випадок, коли розв'язок задачі є невідомим.

Нехай на $\Gamma_2$ задано функції
  \begin{gather*}
	g(x_1, x_2) = x_1-x_2,\\
	q(x_1, x_2) = x_1 +x_2.
 \end{gather*}
 
 \begin{figure}[h!]
\centering
	\vspace*{-4.5cm}
	\includegraphics[scale=.5]{sample1.pdf}
	\vspace*{-4cm}
\end{figure}

 Параметричне задання області має вигляд
  \begin{equation*}
 \begin{split}
 	&\Gamma_1= \{x(s)=(r(s)\cos(s), r(s)\sin(s)),\ s\in[0,2\pi]\},\\
	&\Gamma_2= \{x(s)=(2\cos(s),2\sin(s)),\ s\in[0,2\pi]\},
 \end{split}
 \end{equation*}
 де $r(s)=1+0.15\cos(3s)$.
 
 Розв'язок шукатимемо в точці $x=(1, -1.5)\in \Omega$.
 
\begin{center}
\begin{tabular}{ |c|c|c| } 
\hline
      \multicolumn{3}{|l|}{\quad\quad\quad\quad\quad\quad\quad\quad\quad\quad\quad\quad\quad\quad $\tilde{u}(x)$} \tabularnewline
 \hline
 m & \shortstack{$A_0=A_1=A_2=1, \nu=0.5$}  & \shortstack{$A_0=0.1, A_1= 1,A_2=0, \nu=0.9$}  \\ 
 \hline
 4 & 2.1772 & 1.9093 \\ 
 8 & 2.2293 & 1.9261 \\ 
16 & 2.2374 & 1.9296 \\ 
32 & 2.2369 & 1.9295 \\ 
64 & 2.2369 & 1.9295 \\ 
128 & 2.2369 & 1.9295 \\
 \hline
\end{tabular}
 \captionof{table}{Наближення розв'язку при збільшенні $m$ і зміні основних параметрів}
 \end{center}

\subsection{Приклад 2}
Розглянемо крайову задачу з точним розв'язком
$$u(x)=G(x,y^*), \ x=(x_1, x_2)\in\Omega,$$
де $G(x,y^*)$ - звуження фундаментального розв'язку бігармонійного рівняння, $y^*\notin\Omega$.

Функції, що задані на $\Gamma_2$, мають вигляд

 \begin{gather*}
	g(x) = \frac{\partial G(x,y^*)}{\partial n},\\
	q(x) = M_xG(x,y^*).
 \end{gather*}

\begin{figure}[h!]
\centering
	\vspace*{-4cm}
	\includegraphics[scale=.5]{sample2.pdf}
	\vspace*{-4cm}
\end{figure}

 Параметричне задання області має вигляд
  \begin{equation*}
 \begin{split}
 	&\Gamma_1= \{x(s)=(3\cos(s) + 2\cos(2s) - 0.75, 3\sin(s)),\ s\in[0,2\pi]\},\\
	&\Gamma_2= \{x(s)=(2\cos(s)+\cos(2s),1.5\sin(s)),\ s\in[0,2\pi]\}.
 \end{split}
 \end{equation*}
 
 Розв'язок шукатимемо в точці $x=(0, -2)\in \Omega$.
 
\begin{center}
\vspace*{0.3cm}
\begin{tabular}{ |c|c| } 
\hline
 & $|\tilde{u}-u_{ex}|$ \\
 \hline
 m & \shortstack{$A_0=A_1=A_2=1, \nu=0.5$}  \\
 \hline
 4 & 0.00100300  \\ 
 8 & 0.00001130  \\ 
16 & 7.0444e-08  \\ 
32 & 5.4863e-11 \\ 
64 & 6.5824e-15 \\ 
128 & 6.5824e-17 \\
 \hline
\end{tabular}
\captionof{table}{Абсолютна похибка при збільшенні $m$}
\end{center}

\subsection{Приклад 3}

Нехай точний розв'язок має вигляд
$$u(x)=x_1^2-x_2^2, \ (x_1, x_2)\in\Omega.$$

Тоді крайові функці на $\Gamma_2$ можна обчислити як

 \begin{gather*}
	g(x) = \frac{\partial u(x)}{\partial n},\\
	q(x) = M_xu(x).
 \end{gather*}

Розглянемо наступну область

  \begin{equation*}
 \begin{split}
 	&\Gamma_1= \{x(s)=(3\cos(s) + 2\cos(2s) - 0.75, 3\sin(s)),\ s\in[0,2\pi]\},\\
	&\Gamma_2= \{x(s)=(2\cos(s)+\cos(2s),1.5\sin(s)),\ s\in[0,2\pi]\}.
 \end{split}
 \end{equation*}
 
\begin{figure}[h!]
\centering
	\vspace*{-4cm}
	\includegraphics[scale=.5]{sample3.pdf}
	\vspace*{-4cm}
\end{figure}

 Розв'язок шукатимемо в точці $x=(1, 1)\in \Omega$.
 
\begin{center}
\begin{tabular}{ |c|c| } 
\hline
 & $|\tilde{u}-u_{ex}|$ \\
 \hline
 m & \shortstack{$A_0=A_1=A_2=1, \nu=0.5$}  \\
 \hline
 4 & 0.0009048  \\ 
 8 & 0.0000119  \\ 
16 & 3.0234e-06  \\ 
32 & 1.4993e-07 \\ 
64 & 2.5824e-09 \\ 
128 & 1.3324e-14 \\
 \hline
\end{tabular}
\captionof{table}{Абсолютна похибка при збільшенні $m$}
\end{center}

Отже, очевидно, що вибір допоміжних констант $A_0,A_1,A_2$ впливає на наближення розв'язку. Також вибраний метод для чисельного розв'язування прямої задачі показує необхідну збіжність.

\chapter*{Висновки}
\addcontentsline{toc}{chapter}{Висновки}

\quad В роботі було розглянуто чисельне розв'язування оберненої крайової задачі для бігармонійного рівняння у випадку двозв'язної області методом нелінійних інтегральних рівнянь, що ґрунтується на потенціалах. Було розроблено ітераційний процес для наближеного розв'язування отриманих інтегральних рівнянь. Знайдено похідні Фреше відповідних операторів для лінеаризації рівняння. Для повної дискретизації був застосований метод тригонометричних квадратур. Отриману некоректну систему лінійних рівнянь було розв'язано метод найменших квадратів із застосуванням регуляризації Тихонова.

Була показана розв'язність прямої крайової задачі методом інтегральних рівнянь. Отримана система iнтегральних рiвнянь була параметризована з видiленням особливостей на межах областi. До iнтегральних рiвнянь з видiленою особливiстю був застосований метод Нистрьома, який базується на використаннi тригонометричних формул з вiдокрем- леннмя вагових функцiй. В результатi ми отримали дискретну систему лiнiйних рiвнянь, з якої були знайденi невiдомi коефiцiєнти додаткової функцiї в рiвняннi для знаходжен- ня наближеного розв’язку. Пiсля того ми провели такий самий процес параметризацiї i дискретизацiї для формули пошуку розв’язку, отримали повнiстю дискретний метод для розв’язування задачi.

Були проведенi чисельнi експерименти, які демонструють коректність прямої задачі, а також експоненційну збіжність вибраного методу.

\begin{thebibliography}{99}
 \addcontentsline{toc}{chapter}{Література}

\bibitem{chapko} 
R. Chapko and B. T. Johansson, Integral equations for biharmonic data completion, Inverse Problems and Imaging, Vol. 130 : P.1-12 (2019)
 
\bibitem{NonLinearIE} 
 R. Chapko, V. Vavrychuk and O. I. Yaman, On the non-linear integral equation method for the reconstruction of the inclusion in the elastic body, Journal of Numerical and Applied Mathematics, Vol. 122 : P.1-17 (2019)
 
\bibitem{Uniqeness} 
R. Kress, Inverse dirichlet problem and conformal mapping, Mathematics and Computers in Simulation, P.255-265 (2004)
 
\bibitem{Holmgren} 
H. Hedenmalm, On the uniqueness theorem of Holmgren, Mathematische Zeitschrift, Vol. 281, P.2 (2013)

\bibitem{ChapkoNonLinearIE} 
R. Chapko, On a hybrid method for shape reconstruction of a buried object in an elastostatic half plane, Inverse Problems and Imaging, 3 (2) : P.199-210 (2009)

\bibitem{KressNonLinearIE} 
R. Kress and S. Meyer, An inverse boundary value problem for the Oseen equation, P.13-15 (1994)
 
\bibitem{kress} 
R. Kress, Linear integral equations, Applied Mathematical Sciences, Third Edition, New York (2014) 

\end{thebibliography}

\end{document}